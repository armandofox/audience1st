\section{One-Time Setup: Read Me First}

Here's the ``quick start'' guide to getting your installation
configured, once you have received the ``green light'' from
Audience1st.  Please follow these steps before trying any of the other
system operations.

It's recommended you follow these steps in the order listed.

\subsection{Configuration Variables}
\label{sec:config-vars}

By default, the system comes with a single customer account whose login
is \verb+admin+ and password is \verb+admin+.  Login using this
information to set the configuration variables.

Configuration variables are a set of values specific to each venue that
are expected to change very rarely (or never) during routine operations.
Hence, there is no built-in link to reach this screen: you must type the
URL \url{http://www.audience1st.com/VENUE/options}, replacing
\emph{VENUE} with the ``shortname'' of your venue as given to you by
A1~Patron Systems.\footnote{The shortname is a URL-friendly version of your
venue's name assigned by A1~Patron Systems; for example, the venue
``Silver Star Theater'' might have the shortname \venue+silverstar+.
Contact A1~Patron Systems if you're not sure of your venue's shortname.}

Most of the options are self-explanatory, but the property
\textbf{External CSS File} deserves separate mention.  \af uses strict
XHTML (Extended HTML) and CSS (Cascading Style Sheets) to control the
visual appearance and layout of its pages.\footnote{Currently CSS is the
  \emph{only} way to change \af's look anad feel.  This is hardly a
  restriction, given the power of CSS; skeptics are invited to examine
  \url{csszengarden.com} to get a feel for just how radically different
  the exact same XHTML can be made to look by changing only the CSS
  stylesheet.} 
While a default ``look and
feel'' is provided, you will probably want to create your own CSS
stylesheet so that the \af pages match the ``look and feel'' of your
venue's main Web site.  You should create this CSS file and host it on
your own venue's site, and supply the full URL to the CSS file (e.g.,
\url{http://www.mylittletheater.org/stylesheets/my_styles.css}) as the
value of this option.  A~CSS~designer can use the
information in section~\ref{sec:creating_css} to style the site.  In the
meantime, you can use the value \verb+/default.css+ (the initial slash
is important) to use \af's default look and feel.

Click Save Changes when
you're done; then, \emph{log out} from the built-in \verb+admin+
account and log back in using the account you setup with Box Office
Manager privilege.

\subsection{Create Administrator Account(s)}

The built-in \verb+admin+ account should be used only for top-level
administrative tasks, not day-to-day tasks.  Instead, you should next
create a Box Office Manager level account for day-to-day
administration, as well as (possibly) additional semi-administrative
accounts for other staff.

\begin{enumerate}
\item Still logged in as \verb+admin+,  use the
``Adding a New Patron'' feature (see section~\ref{sec:addpatron}) to add
a patron account for the system's \emph{real} administrator.  Grant
Box Office Manager privilege to the new user when creating.
\item \textbf{Log out}, and then log back in as the new Box Office
  Manager.
\item Optional: use Add New Patron to create additional accounts for
  other staff.  In general, each account should be created with the
  least privilege level that will allow the user to accomplish her
  assigned tasks.  See section~\ref{sec:privilege-levels} for a
  description of the available levels.
\end{enumerate}

\subsection{Goldstar Events Automatic Processing}

If your venue sells tickets through GoldStar Events, Audience1st can be
setup to automatically intercept will-call lists from GoldStar and
integrate them with your regular lists.  (Note that for this feature to
work, you must \emph{also} make sure that the appropriate voucher types
are setup for the eligible shows; details are in
section~\ref{sec:goldstar-setup}.) 

For this to work, you must arrange for the GoldStar will-call lists to
be routed to the email address \verb+goldstar-VENUE@audience1st.com+,
where \verb+VENUE+ is replaced with your venue name.  There's more than
one way to do this.  The easiest way is to 
set \verb+goldstar-VENUE@audience1st.com+ as the email address  for
will-call lists---the nice folks at \verb+venues@goldstarevents.com+ can
help you with this.  The more robust and recommended way is to direct
GoldStar to send its emails to a designated address at your venue's
domain---say, \verb+goldstar-reports@mytheater.org+---and arrange for
that address to act as a \emph{reflector} (also called an \emph{alias})
that forwards the email to \emph{both} the Audience1st.com address
\emph{and} the personal email address of your House Manager or Managing
Director.  Why do this?  So that if something goes wrong during
automatic will-call processing, a human being
will still have received a human-readable copy of the GoldStar will-call
list.  (GoldStar has been known to change the format of their will-call
list unannounced, which causes automatic processing to break.)

\subsection{Donation Funds}

When donations are recorded by backoffice staff, they can be allotted to
one or more Donation Funds.  By default, the only existing fund is the
General Fund.  (This is also the fund to which online donations made at
the time of ticket purchase always go.)  To create an additional fund,
use this somewhat roundabout procedure:

\begin{enumerate}
\item In the yellow Admin button bar, click \emph{Record Donation}.
\item To the \emph{right} of the Donation Fund dropdown menu, click the
  \emph{Add New} button.
\item Enter the name and optional account code\footnote{Account codes
    are optional and for your use only; in most generated reports,
    account codes are produced for aggregate totals, to facilitate
    exporting the information to your bookkeeping software. Account
    codes can always be added later if they're not specified initially.} 
  for  the new donation fund.
\end{enumerate}

\subsection{Set Up Voucher Types}
\label{sec:setup-vouchertypes}

Most of the existing reports will not be available until at least one
Voucher Type is defined.  Furthermore, when entering new shows and show
dates, you need to associate Voucher Types with each show date.
Therefore the next step is to enter at least
one or two basic Voucher Types.

From the main navigation tabs, select \emph{Voucher Types}.  See
section~\ref{sec:vouchertypes-details} for how to add or modify voucher
types.  You can always add more types later and add newly-defined types
to already-entered performances, but enter one or two basic types now,
possibly including your Subscription vouchers.

\subsection{Set Up Shows and Show Dates}
\label{sec:setup-shows-and-showdates}

Lastly, set up your shows and show dates.  See
section~\ref{sec:shows-and-showdates} for instructions.  To avoid
confusion on the part of your patrons, it's recommended that you enter a
show and its showdates all at once.  That is, do not enter a show
without also entering all its showdates---otherwise, the show will
appear as a choice for salable tickets, but no tickets will appear to be
available since no show dates have been defined for it.

\subsection{Unprotect the Site}

Contact A1 Patron Systems to ``unprotect'' the site (remove the master
password that prevents patrons from getting in).

Congratulations---you're live!

