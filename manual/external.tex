\documentclass{article}
\usepackage{url}
\title{Audience1st Styling and Site Integration Guide}
\author{A1 Patron Systems, Inc.}
\date{April 2008}
\newcommand{\af}{{\em Audience1st}}
\newcommand{\tip}[1]{\vspace{\baselineskip}\fbox{\parbox[t]{0.75\textwidth}{#2}}\vspace{\baselineskip}}

\begin{document}
\maketitle

This document describes how to integrate \af{} with your own venue's Web
site.  Two main subtopics are covered:

\begin{itemize}
\item \textbf{Styling:} How to make the customer experience ``seamless''
  moving between your site and \af.
\item \textbf{Integration:} Elements of and links to \af that can be
  directly integrated and ``embedded'' into your site.
\end{itemize}

In general, most transactions having to do with the ticket purchase flow
and reservation flow must occur on the \af{} site, not your venue's.
This may change to some extent in the future.  Checkout/payment
operations must occur on the \af{} site as well, because of \af's
relationship with the payment processor Authorize.net; this is unlikely
to change in the future.

\section{Styling With CSS}
\label{sec:creating_css}

The patron-facing pages of \af{} are heavily tagged with XHTML to
facilitate styling using CSS (Cascading Style Sheets).  You  need some
CSS experience to do this, or you can hire a professional CSS designer;
we're happy to recommend someone if you  need help.

If you haven't already, download the CSS Styling Kit from the A1 Patron
Systems website (Venue Support area).  It contains templates of all
user-facing XHTML pages that a CSS designer can use to style the look
and feel of these pages to match your venue's site.

Keep the following in mind when developing styling:

\begin{itemize}
\item Once you create a stylesheet, you must publish it on your own
  website and enter its URL into the ``Configuration Options'' of \af.  

\item If your stylesheet uses external assets---embedded images,
  background images, etc.---these must also be served from your website.
  In addition, the URL's to those assets must appear in the stylesheet
  as fully-qualified URL's (i.e. beginning with ``http://'' and
  including the fully-qualified domain name of your site; for example,
  \url{http://www.mylittletheater.com/stylesheets/default.css}. 

\end{itemize}

If you plan to ``do it yourself'', you may find the following resources
helpful.  These are mentioned for information only and their mention
doesn't imply any endorsement by or connection to A1 Patron Systems Inc.

\section{Integration}
\label{sec:external-integration}

\af is designed to integrate with external/existing Web sites in various
ways described in this section.

\subsection{Frames}



\subsection{Don't Use IFrames}
\label{sec:iframes}

An IFrame is an HTML construct that allows you to display content from a
third-party site within a ``frame'' served from your main Web site.

\textbf{Do not attempt to embed Audience1st into your venue's primary
  web page using IFrames.}  The reason is that most browsers are
configured not to honor so-called ``browser cookies'' from sites
rendered within an IFrame, yet without cookies it's impossible for a
patron (or administrator) to login to \af.

Instead, customize \af's look and feel using the CSS hooks provided, as
described in section~\ref{sec:creating_css}.

\subsection{``Donate Now'' Formlet}

A \emph{formlet} is just our cutesey word for a tiny HTML fill-out form
that you can embed anywhere on your Web site.  Submitting the formlet
will cause \af{} to take some action and may also deposit the user on an
\af{} page.

One kind of formlet allows the patron to enter a donation amount (in
whole numbers), click ``Donate Now'', and land on the \af{} Store page
with the entered donation already in her Shopping Cart.  She can then
continue shopping (for tickets, e.g.) or begin the normal checkout flow.

To include a ``donate now'' formlet, add the following code to your HTML
page.  The attributes shown below \emph{must} appear exactly as shown,
but you can add additional attributes to style the formlet:


\section{Links}
\label{sec:linking}

You can place various kinds of links directly on your own Web site to
take patrons to \af pages with preselected choices.  In the following
table, all URL's shown are \emph{relative} to the path
\url{http://www.audience1st.com/VENUE} where VENUE is replaced with the
shortname of your venue.  That is, for the fictitious Silver Star
Theater  whose shortname 
assigned by A1~Patron Systems is \verb+silverstar+, 
a URL shown in the table as
\emph{/store} would expand to
\url{http://www.audience1st.com/silverstar/store}.

\begin{table}
  \begin{tabular}{|l|p{0.75\textwidth}}
    \hline
    \textbf{URL} & \textbf{Where it goes} \\
    \hline
    /customers/login  & Login page for subscribers and returning
    customers \\
    /store & Default page for buying individual tickets \\
    /store?showdate\_id=\emph{id} & Purchase individual tickets with
    performance preselected to the performance whose showdate ID is \emph{id}.
    You can determine the showdate ID by going to the \emph{Shows} tab
    and clicking on the name of a show. \\
    /store?show\_id=\emph{id} & Purchase individual tickets with
    production preselected to the show whose show ID is \emph{id}, but
    no specific performance preselected.  Show
    ID's are shown to the left of each show name when you click the
    \emph{Shows} tab. \\
    /store/subscribe & Page for purchasing subscriptions for the current
    season.  If no such subscriptions are still on sale, patron will
    simply be redirected to the page for regular ticket purchase. \\
  \end{tabular}
\label{table:links}
\caption{External links directly pointing to Audience1st.  Replace
  VENUE with the shortname of your venue in all examples.}
\end{table}

\subsection{Add a Donation Button}

You can create a miniature HTML form to allow direct entry of a
donation.  Use the 

\subsection{Displaying the Cart}



\tip{\textbf{Why can't I place a self-contained ``Add ticket to cart''
    button directly on my venue's page?}  Because the type(s) of tickets
  available to be added to the cart depend on many factors including
  whether the patron is logged in and whether s/he is a subscriber.}

\subsection{RSS Feed}
\label{sec:rss}

\af provides an RSS feed of upcoming performances, where each feed item
includes a one-line description of the performance's sales volume
(``Available,'' ``Nearly sold out,'' or ``Sold out'') and is a link to
the ticket-purchase page preselected for that performance (sold-out
shows don't include a link).

The RSS feed is compatible with most popular content-management or blog
software that includes an RSS feed plug-in.  The feed returns one item
for each upcoming show in the calendar season, so if you want to display
fewer, you'll have to configure your RSS plug-in or widget accordingly.

The URL at which the feed is available is
\url{http://www.audience1st.com/VENUE/info/ticket_rss}.

\subsection{Show Calendar}
\label{sec:calendar}

\af exports the calendar of all upcoming performances for the calendar
season in the standard vCalendar format.  This makes it suitable for
subscribing to the calendar from any plug-in or application (e.g. iCal
on Mac OS~X, Outlook 2003 or later) that can subscribe to a
vCalendar-format calendar.  It also makes it easy to display a ``season
calendar'' on your venue's Web site by using a plug-in that understands
the vCalendar format.

The URL at which the calendar is available is
\url{http://www.audience1st.com/VENUE/info/calendar_ical}.

