\section{Walkup Sales}
\label{sec:walkup}

If you sell tickets at the box office right before showtime, you can use
the Walkup Sales capability.

The Walkup Sales functionality can be accessed by any user whose
privilege level is ``Walkup Sales'' or higher.

The key to understanding the design of the Walkup Sales interface is its
assumption that you \emph{do not have time to collect} identifying
information (i.e. names and addresses) for walkup patrons.  All
purchases recorded through the Walkup Sales interface are allocated to
the \textbf{WALKUP~CUSTOMER}, a fictitious placeholder customer that
cannot be deleted.  Furthermore, walkup transactions are marked to
distinguish them from advance-sales transactions, so that settlement
reports based on night-of-show sales can be easily generated.

Two common scenarios may arise on a show night in which the Walkup Sales
interface is not appropriate:

\begin{enumerate}
\item For whatever reason, you specifically \emph{do} want to tie a
  walkup purchase to a particular individual.  (Perhaps you have a
  loyalty program where patrons who see a certain number of shows get
  one free.)  In this case, you must
  enter it just as you would a phone transaction
  (section~\ref{sec:advance_sales}), presumably while the angry mob
  waits their turn.
\item A Subscriber without an advance reservation wants to redeem a
  Subscriber voucher as a walk-up.  (Your policy on whether to allow
  this and how to handle it may vary.)  In this case, you must visit the
  Subscriber's account and place the reservation just as you would for
  an advance reservation (section~\ref{sec:reservation_details}).  Note
  that Walkup Sales user privilege specifically allows this operation,
  for this exact reason.
  
\end{enumerate}

\subsection{Recording Walkup Sales}
\label{sec:walkup_sales}

When you click the \emph{Walkup Sales} tab, a two-column screen will
appear. At the top of the leftmost column, you can select a performance
date (it defaults to today's date).  Beneath that are dropdown menus for
\emph{every ticket type allowed for walkup sales.}  This is an important
point: if a particular (say, discount) ticket type is valid for only
certain performances of the run, it \emph{will appear} in the list of
choices for walkup sales, and it is \emph{up to the Box Office agent} to
enforce policies regarding nonstandard ticket types.  This is a
deliberate design choice, since it is usually preferable to provide more
leeway and allow humans to make choices, especially in the hectic
atmosphere of night-of-show sales.

To record a walkup purchase, simply select the correct performance date,
then select the number(s) of ticket(s) of each type 
