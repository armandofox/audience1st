
\section{Box Office Manager's Guide}
\label{chap:bom}

Whereas Box Office Agents can handle ticket sales and reservations for
performances, the Box Office Manager is the one who \emph{causes tickets
  and reservations to become available} for shows, by entering
a show into the system, associating performance dates with
it, and indicating what tickets are redeemable for each
performance. These steps are
collectively called \emph{listing the show}.

In general, listing a show involves three steps:

\begin{enumerate}
\item[Add the show] Enter the overall information for the show---name, 
  default seating capacity, etc.  This step can be
  skipped if just adding or removing performance dates from an existing
  show. 
\item[Add show dates] Associate performance dates with the show (or
  remove performance dates).  Information such as the deadline for
  advance sales cutoff and the specific house capacity can be overridden
  for each show date.  The show's ``opening date'' and ``closing date''
  are automatically determined from the earliest and latest showdates.
\item[Indicate valid vouchers] Associate particular ticket types
  (General Admission, Subscriber, etc.) with the performance.
  \emph{Without this step, the performances are listed but no vouchers
    can be redeemed or tickets sold for those performances.}
\end{enumerate}

In addition, the Box Office Manager can define new types of vouchers
(tickets) such as discount tickets, promotions, limited-capacity
tickets, etc.  In addition to \emph{regular} vouchers, the Box Office
Manager can set up \emph{bundle} vouchers; a bundle is a collection of
single tickets that is offered to the customer as a unit, for example, a
subscription or series package.

Before listing any shows, it makes sense to define some \emph{voucher
  types}, which correspond to the types of tickets your venue sells,
including both revenue (``single tickets'') and subscription vouchers.
The next section covers how to set up both regular and bundle vouchers.
In general, you would set up most basic voucher types once only, and
only occasionally add new voucher types when you want to add a new
promotion or ticket type.
The subsequent three sections cover the three steps in listing a show:
entering show information (section~\ref{sec:addingshows}), adding
performance dates for a show (section~\ref{sec:addingshowdates}), and
indicating voucher validity for each performance date
(section~\ref{sec:validvouchers}).  

\subsection{Vouchers and Bundles}
\label{sec:vouchertypes-details}

There are two types of vouchers.  A \emph{regular} voucher is a single
seat or ticket to a single performance with a price of zero dollars or
more.  For example, a General Admission seat would be a regular voucher,
as would a Staff Comp.  A \emph{bundle} is a package
that includes several regular vouchers.  A subscription is usually one
type of bundle, but nonsubscription bundles are also possible; for example,
a 2-for-1 promo or kid-free-with-adult promo.
To create a bundle, you first have to create the regular vouchers the
bundle will contain, so we'll cover that first.

\subsubsection{Creating a Regular Voucher Type}
\label{sec:addingregularvouchertypes}

When a new voucher type is set up, 
the following properties must be specified for it:

\begin{itemize}
\item[Name] The name that will appear (e.g. in dropdown menus on
  purchase screens, in reports, etc.) for the voucher; for example,
  ``Adult General Admission'' or ``Staff Comp''.  Note that
  you can change the name at any time in the future without affecting
  anything except the appearance of menus and reports.
\item[Price] The price of this type of voucher.  Enter zero for comps.
  Don't enter a negative number.  When a voucher is sold, the sale price
  is recorded as the value of this property.  Note that if you change
  the sale price of a voucher type, the change applies retroactively to
  all purchases.    So, if you expect
  to raise General Admission prices, you will need to create a new
  Voucher Type for the different price in the future.
  \emph{future} purchases (as you'd expect).  (This will change soon,
  allowing the price paid for vouchers to remain the same even after the
  voucher type's price is changed.)
\item[Account Code] If you use accounting software, you can enter a code
  here (up to 8 characters) corresponding to the account code you use to
  track sales of this voucher type.  It's OK for multiple voucher types
  to share an account code; for example, you may have different voucher
  types for Adult vs. Student general admission, yet the income from
  both types should be aggregated under the same account code.  The
  account code is used in the Accounting Reports, which break out
  revenue by production and account number.  Obviously, voucher types
  whose price is zero need not have an account code.  You can leave this
  blank initially and fill it in later, and your reports will still be
  correct (i.e., changing the account code is retroactive to all
  vouchers already sold of a particular type).
\tip{Important}{If you give \emph{any} voucher type an account code, to
  be consistent you should give \emph{all} voucher types account codes.
  Otherwise, when generating the accounting report, any voucher types
  with no account code will silently be excluded from the report.}
\item[Not valid before] The earliest date on which a voucher of this
  type can be purchased or redeemed. This allows you to define voucher
  types far in advance of when they will be offered for sale.
\item[Not valid after] A date after which the voucher expires and can no
  longer be used.  Useful for time-limited promotions.  When a voucher
  of a particular type is issued, it ``inherits'' this expiration date.
  However, if the Voucher Type's expiration date is changed, the change
  is \emph{not} applied to Vouchers of that type already issued.

\tip{Example}{The ``General Admission'' voucher type is defined with
an expiration date of December 31, 2008.  All customers
purchasing General Admission tickets will get vouchers that expire on
that date.  When the new season rolls around, the General Admission
voucher type's expiration date is changed to December 31, 2009.  This
later date will apply to any \emph{future} vouchers of this type that
are issued, but customers already holding vouchers of this type will
still have the old expiration date.}

\item[Mail fulfillment needed] Check this box if acquisition of the
  voucher by a patron implies manual fulfillment.  This is used by the
  ``Generate Orders Needing Fulfillment'' report.  You might set this,
  e.g., for Subscription bundles where you prefer to mail the customer
  some collaterals.  (Note that this doesn't affect the customer's
  ability to use the voucher---she can make reservations as soon as the
  voucher is paid for.)
\tip{Important}{When a bundle voucher is marked as requiring
  fulfillment, \emph{only} the actual bundle voucher itself---and not
  the regular vouchers included with it---will appear on the ``Orders
  Needing Fulfillment'' report
  (section~\ref{sec:report_unfulfilled}). This is usually what you want,
  because (e.g.) in the case of a family that buys 2 subscriptions
  containing 4 tickets each, you will get 2 lines on the fulfillment
  report rather than $2+(2\times 4)=10$ lines.}
\item[Walkup sales allowed] Some vouchers can also be sold at the door,
  e.g. General Admission.  Others, such as advance-purchase-only
  specials, cannot.  Only voucher types that have this flag set will
  appear on the Walk-Up Sales screen (section~\ref{sec:walkup}).
\tip{Note}{You \emph{should not} allow Subscription
  Bundles to be sold as Walkup, because the Walkup Sales interface does
  not allow capturing the patron's name and address.  Use this field
  only for voucher types that you're comfortable selling
  ``anonymously''.}

\item[Is bundle] To create a bundle voucher, you must first create all
  of the individual voucher types that the bundle will contain,
  \emph{then} create a new Bundle voucher type, then edit it to include
  the right quantity of each voucher.  That's covered in the next section.
\item[Is subscriber] Does the purchase of this voucher qualify the
  purchaser as a ``subscriber''?  Useful for distinguishing bundles
  corresponding to subscriptions from bundles corresponding to promos.
  Note that while you probably only want to set this property for Bundle
  vouchers, a non-bundle voucher can also be designated as
  subscriber-qualifying. 
\item[Changeable] Specifies whether the voucher is tied to a particular
  show date or can be used to reserve for a variety of dates. This
  attribute is not directly settable by the BOM; it's set by the system
  when the voucher is created.  Specifically, vouchers that are
  pre-purchased as part of subscriptions or promos are generally
  changeable; vouchers corresponding to tickets purchased for a
  particular performance are generally not.
\item[Comments] Any additional comments, seen only by staff when working
  with reports.
\end{itemize}


\subsubsection{Adding a New Bundle Voucher Type}
\label{sec:addingbundlevouchertypes}


A bundle is a voucher that cannot itself be ``redeemed'' for anything
but serves as a placeholder.  For example, if a patron buys a
subscription containing 3 vouchers good for any performance, they would
actually see 4 vouchers in their account.  The first is a nonredeemable
``voucher'' that records the type of bundle purchase.  The remaining 3
are the actual vouchers that can be redeemed for seats.  This makes it
easy to figure out how many people took advantage of a particular
bundle promotion even if the promotion includes ``regular'' tickets.

In general, to define a bundle voucher, you \emph{first} have to define the
regular vouchers that the bundle is going to include, as described
above.   Then create a new voucher type and check the ``Is Bundle'' box,
and save the voucher.  Next click the \emph{Edit} link next to the
newly-created bundle voucher's name, and you will see an ``Included
Vouchers''  panel where you can enter the
number of each type of regular vouchers included in the bundle.

\subsection{Approaches to Subscriptions}
\label{sec:subscriptions}

In general, a subscription is just a bundle that qualifies the buyer as
a Subscriber.  The bundle includes some number of ``vouchers'' each of
which can be used to make a reservation for a season show.  However,
there's many ways to set up subscriptions, and it's worth understanding
the differences before you set one up.

For the purposes of these examples, we'll assume you have a 4-show
season in which you present 2 plays and 2 musicals: Deathtrap, Chicago,
The Foreigner, and Seussical, in that order.

\subsubsection{One of Each Show}

The most conventional subscription type gives the subscriber exactly one
reservation for each production.  In other words, the subscriber cannot
choose to use (say) all 4 subscription vouchers to get 4 reservations
for the same performance or even for the same production.

To set this up, you'd first create 4 different types of regular
vouchers, with names such as  \emph{Subscriber - Chicago}, \emph{Subscriber -
Seussical}, \emph{Subscriber - Deathtrap}, and \emph{Subscriber -
Foreigner}.  When you list the shows, you will arrange that only
\emph{Subscriber - Chicago} vouchers can be redeemed for Chicago, etc.  You
then create a bundle \emph{2008 Subscriber} that includes one each of
those four vouchers.

In fact, after Deathtrap closes, you could further
define another bundle \emph{Late Season} that includes only the 3
remaining shows.

Note that if a single subscriber purchases multiple subscriptions
(e.g. a family of 4), the subscription vouchers don't all have to be
redeemed together---that is, they can choose to all attend the same show
or four different performances.

\subsubsection{Limited Flex}

Another type of subscription is ``Limited Flex''.  For example, you
might separate your shows into categories (say, plays vs. musicals, or
dramas vs. comedies) and then setup a voucher good for any show in that
category.  You could then offer, e.g., a ``two from group A and one from
group B'' mini-subscription, and the two vouchers for group A could be
used either for two different productions or for two seats to the same
production. 

\subsubsection{Flex}

The ``flex pass'' type of subscription has been gaining some popularity:
the patron receives a fixed number of vouchers to be redeemed however
she wants---all at once for a single production, one for each
production, etc.  Typically these are priced more expensively than
regular ``one of each'' subscriptions but still priced to offer a break
compared to buying single tickets.

Lastly, a caveat.  It's easy to get carried away with all the
flexibility, and having too many types of subscriptions seems to confuse
some patrons.  Think about one or two easy-to-explain scenarios and
stick to them.

TBD: discuss connection between how subscription is setup and whether to
use Classic View or Streamlined(?) View for subscriber welcome screen

\subsection{Shows and Show Dates}
\label{sec:shows-and-showdates}

A \emph{show} is a series of 1 or more performances of the same piece.
In general, to list a show you must first enter the general information
about the production (opening and closing dates, house capacity, etc.),
then enter specific information about each show date.

Several default properties apply to the show (production) as a whole,
such as the capacity of the house, but can be overridden selectively for
individual dates (for example, if you want to hold back house seats on
opening night of a performance, you can set this up automatically to
reduce the number of tickets allowed to be sold for that performance).

\label{sec:addingshows}
\label{sec:addingshowdates}

The next step is to add show dates for the production.  To do this,
click on the show's name in the list of all productions (you can get to
this list by clicking the Shows navigation tab), and then click
the \emph{Add A Performance} button on the Show Details page.

For a new performance, you need to enter these items:
\begin{enumerate}
\item[Date and Time] The curtain date and curtain time.  This
  information is also used by the External Integration features such as
  the calendar and RSS feed (see section~\ref{sec:external-integration}.
\item[Advance sales stop] When do online sales for this performance
  stop.  As you will see, you can also set start/end sales dates for
  different ticket types for the same performance, but \emph{no} online
  tickets will be sold after the time specified here, regardless of
  their individual End Sales deadlines.  Note that a Box Office Agent
  processing phone sales can override this deadline, as do Walkup
  Sales (section~\ref{sec:walkup}) processed at the door.
\item[Max sales] Enter a number here to restrict the total number of
  tickets sold to be something \emph{other} than the house capacity
  specified for the show as a whole (e.g. to hold back a block of seats
  to be purchased through a separate channel or as house seats).  Enter
  0 to make the max sales the same as the house capacity.
\end{enumerate}

\tip{Note}{The External Integration Tools can be made to actually report
  a show as ``Sold Out'' \emph{before} the maximum number of tickets
  have been sold.}

\subsection{Managing Voucher Validity For Show Dates}

To allow a particular type of voucher (e.g. ``General Admission'') to be
redeemed for a particular performance, you must indicate that it is a
\emph{valid voucher} for that performance.

\tip{Important.}{When a voucher type is first created, it is
\emph{not} automatically redeemable for any particular show.  Its
validity for specific performances \emph{must} be explicitly established as
described in this section.}

Note that a bundle voucher cannot be made redeemable, but the regular
vouchers included in the bundle can be.  For example, if a \emph{Family
Promo} bundle voucher includes 2 \emph{Adult} and 2 \emph{Child} tickets, the
ticket types \emph{Adult} and \emph{Child} can be made redeemable, but the
bundle \emph{Family Promo} cannot be.

\subsubsection{Adding Valid Vouchers to a Performance}
\label{sec:addingvalidvouchers}

Adding valid voucher types for a performance makes those voucher types
available for purchase or reservation for that performance.

To make a particular regular (nonbundle) voucher type valid for
redemption to a given performance, first be sure that the voucher type
exists (see section~\ref{sec:addingregularvouchertypes}) and the show
date for which you want to allow redemption exists (see
section~\ref{sec:addingshowdates}).  Navigate to the list of show dates
for the show in question (section~\ref{sec:addingshowdates}.

Next to each show date is the list of voucher types currently redeemable
for that performance.  (See figure XXX.)  If the voucher type you want
to make valid for the performance isn't listed there, click \emph{Add
  New}.  (If you \emph{do} see the voucher type listed, but you want to
change any validity properties such as redemption limits or sales
cutoff, click on the name of the voucher type and see
section~\ref{sec:changingvalidvoucher}.) 

You must specify the following redemption properties for the voucher:

\begin{itemize}
\item[Voucher type] Select from the dropdown list of voucher types
  you've defined.  Only non-bundle vouchers will appear in the list.
\item[Max sales] To limit capacity of this ticket type, enter a number
  here; for example, you may want to limit Staff Comps redemption to no
  more than 4 per performance, or you may want to limit the number of
  discounted tickets sold per performance. Enter 0 for unlimited (i.e.,
  up to the capacity of the house).
\item[Password] If you enter a password, patrons buying online will be
  required to enter it (``promo code'') to be able to buy that ticket
  type.  You can actually enter multiple promo codes separated by commas
  if you want to track redemption separately for the same promo offered
  through different ad channels.
\item[Start sales on] Enter the earliest date and time at which this
  type of ticket goes on sale.
\item[End sales] You can either end sales a certain number of hours
  (decimal points OK) before curtain, or on a specific date and time.
\item[Add ticket type to] Specify whether this valid voucher should be
  added only for the specific performance or for all performances. (In
  the future you will be able to select any arbitrary subset of
  performances to add to.)
\end{itemize}

\tip{Important}{If you specify ``Add to All Dates'', you probably want
  to specify the end sales as ``hours before
  curtain'', which will then be
  correctly computed for each show date.  If you specify the end sales
  as a specific date, that \emph{same} date will be the end-sales date
  for \emph{all} show dates to which it's added.}


\subsection{Ticketing Strategy Tips}
\label{sec:ticketingstrategy}

If you have a ``signature'' show that's expected to anchor your season,
you could define the Full (regular) subscription to include that show,
but any Short or Late Season subscriptions to exclude it, to maximize
your subscriber base.

Using the Advance Sales Start date when setting up valid vouchers allows
you to offer benefits to subscribers such as earlier reservations for
shows that may sell out.

Using the Advance Sales Stop date allows you to setup time-limited
promotions, e.g. ``Purchase your tickets by midnight Tuesday for a
discount on Friday's show.''  You would implement this by creating a
special Voucher Type for the discounted ticket (e.g. \emph{Early Bird
  Discount}) and set it up as redeemable for certain performances but
with advance sales start/stop dates that are more restrictive than those
for your ``regular'' general admission.

Remember that Subscribers (by definition, anyone who acquires a voucher
that has the Is Subscription flag set) are visually identified on the
Box Office report (door list) as well as in all displayed customer
listings, making it easy to single them out for special treatment.

